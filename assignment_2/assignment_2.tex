\documentclass[a4paper, 12pt]{article}

\title{Game Theory I - Problems Set 1}
\author{Baris Sirin - 703490}

\begin{document}
	
\maketitle
\pagenumbering{gobble}
\newpage
\pagenumbering{arabic}
	
	\textbf{Problem 1}\\

	\textbf{Question a}\\ 
	\\
	 Inverse demand function with $ P$ as the market clearing price:\\
	 
	$P(Q)=a-Q (with$ $ a>0)$\\
	\\	
	Each firm has marginal costs but no fixed costs: $MC=c (with$ $c <a )$\\
	\\
	Aggregate supply equals: $q=q_{1}+...+q_{n}$\\
	\\
	Strategy sets: $S_{i} = [0,\infty), i=1,...,n$\\
	\\
	Aggregate supply of all other firms except firm 1: $Q_{-1}=q_{2}+...+q_{n}$\\
	\\
	Residual demand for firm 1 is: $p= (a-Q_{-1} - q_{1})$\\
	\\
	Payoff function for firm 1 is: $\pi_{1}(q_{1}, Q_{-1})=((a-Q_{-1}-q_{1}\times q_{1})-(c\times q_{1})$\\
	\\
	If $Q_{-1} \leq a-c$ the best response for firm 1, derived from the FOC, would be:\\
	
	$\frac {\partial \pi_{i} (q_{1}, Q_{-1})} {\partial q_{1}}=a-Q_{-1}-2q_{1}-c$\\
	\\
	If we equate the FOC to 0 and rewrite, we get the following: \\
	
	$q_{1}=\frac{1}{2}\times(a-Q_{-1}-c)$\\
	\\
	SOC gives us that the best response function is concave and has a max: \\
	
	$\frac{\partial^2\pi_{i}(q_{1}, Q_{-1}}{\partial q_{1}^2}=-2<0$\\
	\\
	If $Q_{-1}>a-c$ we find that: \\
	
	$\pi_{1}(q_{1}, Q_{-1}9=((a-Q_{-1}-q_{1})\times q_{1})-(c\times q_{1}) < 0$\\
	\\
	Since each firm is identical (rewriting the FOC)
	
	$2q_{1} + Q_{-1} = a-c$	
	
	$2q_{cournot}=a-c$	
	
	$q_{cournot}=\frac {(a-c)}{(n+1)}$
	\\
	
	\textbf{Question b}\\
	
	As the number of firms increases, Cournot equilibrium output increases toward the long-run competitive equilibrium output level.\\
	\\
	Since the inverse demand function with P as the market clearing price:\\
	
	$P(Q))a=Q (with a>0)$\\
	\\
	Q is in this case $(n\times q_{cournot})$\\
	
	$\Delta(n\times q_{cournot})>\Delta \frac {(a-c)}{(n+1)}$\\
	\\
	And if Q is max (n to infinity), P(Q) is minimized with respect to Q\\
	
	$\lim_{N\to \infty} p=c$\\
	
	At this level the equilibrium price falls towards the constant marginal cost. This is in line with the Bertrand equilibrium. The quantitiy produced and number of suppliers are also equal to the Bertrand and perfectly competitive level (no fixed or other entry assumed)\\
	\\
	\textbf{Problem 2}\\
	\\
	\textbf{Question a}\\
	
	Customer at location x is indifferent between buying \textit{I} and buying \textit{U}. Then utilities from buying both vendors are equal for this customer. Therefore,\\
	
	$ u_{x}(I)=v-P_{I}-x$ and $u_{x}(U)=v-p_{U}-(1-x)$\\
	
	$v-P_{I}-x=v-P_{U}-(1-x)$\\
	\\
	If we solve this equation for x, we can acquire demand functions for both vendors, which is:\\
	
	$x=\frac{1+P_{U}-P_{I}}{2}$\\
	\\
	Thus the demand for \textit{I}:\\
	
	$Q_{U}(P_{I}, P_{U}) = 72x = 72\left (\frac{1+P_{U}-P_{I}}{2}\right)$\\
	
	$Q_{U}(P_{I}, P_{U}) = 72$ $(1-x) = 72\left (1- \left (\frac{1+P_{U}-P_{I}}{2}\right)\right)$\\
	\\
	Profit functionf for \textit{I-Scream} is:\\
	
	$\pi_{I}(P_{I}) = P_{I} \times Q_{I} = P_{} \times 72\left (\frac{1+P_{U}-P_{I}}{2}\right)$\\
	\\
	FOC of $\pi_{I}$:\\

	$\frac{\partial \pi_{I}(P_{I})}{\partial P_{I}}=72 + 72P_{U} - 144P_{I} = 0$\\
	\\
	And we found the best response function for \textit{I-scream} as :\\
	
	$P_{I}=\frac{1+P_{U}}{2}$\\
	\\
	Likewise, profit function for \textit{U-scream} is,\\
	
	$\pi_{U}(P_{U}) = P_{U} \times Q_{U} = P_{U} \times 72 \left (1 - \left (\frac {1 + P_{U} - P_{I}}{2} \right) \right)$\\
	\\
	FOC of $\pi_{U}$:\\

	$\frac{\partial \pi_{U}(P_{U})}{\partial P_{U}}=72 + 72P_{I} - 144P_{U} = 0$\\
	\\
	And we found the best response function for \textit{U-scream} as :\\
	
	$P_{U}=\frac{1+P_{I}}{2}$\\
	\\
	If we plug in the best response function of one vendor into other vendor’s best response function, we can find Nash Equilibrium between these two vendors as,
	
	$P_{I} = P_{U} = 1$, and $x = \frac {1}{2}$
	\\
	Finally the profits of the \textit{I-scream} and \textit{U-scream} are,
	
	$\pi_{I} (P_{I}) = P_{I} \times Q_{I} = 72 \times \frac {1}{2} = 36$\\
	$\pi_{U} (P_{U}) = P_{U} \times Q_{I} = 72 \times \left (1- \frac {1}{2} \right ) = 36$
	
\newpage

		\textbf{Question b}\\
		
		If both vendors at the same location, this means all customers become indifferent between two vendors with respect to vendor location. So vendors can only compete with their prices. Vendor with lower price will get whole demand. \\
		
		Since all customers are indifferent between two vendors, utility of buying from I-scream is equal to utility of buying from U-scream, Nash equilibrium is,\\
		
		$u_{i}(I) = v - P_{I} - x$ and $ u_{i}(U) = v - P_{u} - x$\\
		
		$v - P_{I} - x = v - P_{U} - x$\\
		
		$P_{I} =  P_{U} = c = 0$\\
		\\
		Therfore, \\
		
		$\pi_{I} = \pi_{U} = 0 $
		
	
\end{document}
